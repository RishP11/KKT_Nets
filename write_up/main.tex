\documentclass{article} % For LaTeX2e
\usepackage{iclr2025_conference,times}

% Optional math commands from https://github.com/goodfeli/dlbook_notation.
%%%%% NEW MATH DEFINITIONS %%%%%

\usepackage{amsmath,amsfonts,bm}

% Mark sections of captions for referring to divisions of figures
\newcommand{\figleft}{{\em (Left)}}
\newcommand{\figcenter}{{\em (Center)}}
\newcommand{\figright}{{\em (Right)}}
\newcommand{\figtop}{{\em (Top)}}
\newcommand{\figbottom}{{\em (Bottom)}}
\newcommand{\captiona}{{\em (a)}}
\newcommand{\captionb}{{\em (b)}}
\newcommand{\captionc}{{\em (c)}}
\newcommand{\captiond}{{\em (d)}}

% Highlight a newly defined term
\newcommand{\newterm}[1]{{\bf #1}}


% Figure reference, lower-case.
\def\figref#1{figure~\ref{#1}}
% Figure reference, capital. For start of sentence
\def\Figref#1{Figure~\ref{#1}}
\def\twofigref#1#2{figures \ref{#1} and \ref{#2}}
\def\quadfigref#1#2#3#4{figures \ref{#1}, \ref{#2}, \ref{#3} and \ref{#4}}
% Section reference, lower-case.
\def\secref#1{section~\ref{#1}}
% Section reference, capital.
\def\Secref#1{Section~\ref{#1}}
% Reference to two sections.
\def\twosecrefs#1#2{sections \ref{#1} and \ref{#2}}
% Reference to three sections.
\def\secrefs#1#2#3{sections \ref{#1}, \ref{#2} and \ref{#3}}
% Reference to an equation, lower-case.
\def\eqref#1{equation~\ref{#1}}
% Reference to an equation, upper case
\def\Eqref#1{Equation~\ref{#1}}
% A raw reference to an equation---avoid using if possible
\def\plaineqref#1{\ref{#1}}
% Reference to a chapter, lower-case.
\def\chapref#1{chapter~\ref{#1}}
% Reference to an equation, upper case.
\def\Chapref#1{Chapter~\ref{#1}}
% Reference to a range of chapters
\def\rangechapref#1#2{chapters\ref{#1}--\ref{#2}}
% Reference to an algorithm, lower-case.
\def\algref#1{algorithm~\ref{#1}}
% Reference to an algorithm, upper case.
\def\Algref#1{Algorithm~\ref{#1}}
\def\twoalgref#1#2{algorithms \ref{#1} and \ref{#2}}
\def\Twoalgref#1#2{Algorithms \ref{#1} and \ref{#2}}
% Reference to a part, lower case
\def\partref#1{part~\ref{#1}}
% Reference to a part, upper case
\def\Partref#1{Part~\ref{#1}}
\def\twopartref#1#2{parts \ref{#1} and \ref{#2}}

\def\ceil#1{\lceil #1 \rceil}
\def\floor#1{\lfloor #1 \rfloor}
\def\1{\bm{1}}
\newcommand{\train}{\mathcal{D}}
\newcommand{\valid}{\mathcal{D_{\mathrm{valid}}}}
\newcommand{\test}{\mathcal{D_{\mathrm{test}}}}

\def\eps{{\epsilon}}


% Random variables
\def\reta{{\textnormal{$\eta$}}}
\def\ra{{\textnormal{a}}}
\def\rb{{\textnormal{b}}}
\def\rc{{\textnormal{c}}}
\def\rd{{\textnormal{d}}}
\def\re{{\textnormal{e}}}
\def\rf{{\textnormal{f}}}
\def\rg{{\textnormal{g}}}
\def\rh{{\textnormal{h}}}
\def\ri{{\textnormal{i}}}
\def\rj{{\textnormal{j}}}
\def\rk{{\textnormal{k}}}
\def\rl{{\textnormal{l}}}
% rm is already a command, just don't name any random variables m
\def\rn{{\textnormal{n}}}
\def\ro{{\textnormal{o}}}
\def\rp{{\textnormal{p}}}
\def\rq{{\textnormal{q}}}
\def\rr{{\textnormal{r}}}
\def\rs{{\textnormal{s}}}
\def\rt{{\textnormal{t}}}
\def\ru{{\textnormal{u}}}
\def\rv{{\textnormal{v}}}
\def\rw{{\textnormal{w}}}
\def\rx{{\textnormal{x}}}
\def\ry{{\textnormal{y}}}
\def\rz{{\textnormal{z}}}

% Random vectors
\def\rvepsilon{{\mathbf{\epsilon}}}
\def\rvtheta{{\mathbf{\theta}}}
\def\rva{{\mathbf{a}}}
\def\rvb{{\mathbf{b}}}
\def\rvc{{\mathbf{c}}}
\def\rvd{{\mathbf{d}}}
\def\rve{{\mathbf{e}}}
\def\rvf{{\mathbf{f}}}
\def\rvg{{\mathbf{g}}}
\def\rvh{{\mathbf{h}}}
\def\rvu{{\mathbf{i}}}
\def\rvj{{\mathbf{j}}}
\def\rvk{{\mathbf{k}}}
\def\rvl{{\mathbf{l}}}
\def\rvm{{\mathbf{m}}}
\def\rvn{{\mathbf{n}}}
\def\rvo{{\mathbf{o}}}
\def\rvp{{\mathbf{p}}}
\def\rvq{{\mathbf{q}}}
\def\rvr{{\mathbf{r}}}
\def\rvs{{\mathbf{s}}}
\def\rvt{{\mathbf{t}}}
\def\rvu{{\mathbf{u}}}
\def\rvv{{\mathbf{v}}}
\def\rvw{{\mathbf{w}}}
\def\rvx{{\mathbf{x}}}
\def\rvy{{\mathbf{y}}}
\def\rvz{{\mathbf{z}}}

% Elements of random vectors
\def\erva{{\textnormal{a}}}
\def\ervb{{\textnormal{b}}}
\def\ervc{{\textnormal{c}}}
\def\ervd{{\textnormal{d}}}
\def\erve{{\textnormal{e}}}
\def\ervf{{\textnormal{f}}}
\def\ervg{{\textnormal{g}}}
\def\ervh{{\textnormal{h}}}
\def\ervi{{\textnormal{i}}}
\def\ervj{{\textnormal{j}}}
\def\ervk{{\textnormal{k}}}
\def\ervl{{\textnormal{l}}}
\def\ervm{{\textnormal{m}}}
\def\ervn{{\textnormal{n}}}
\def\ervo{{\textnormal{o}}}
\def\ervp{{\textnormal{p}}}
\def\ervq{{\textnormal{q}}}
\def\ervr{{\textnormal{r}}}
\def\ervs{{\textnormal{s}}}
\def\ervt{{\textnormal{t}}}
\def\ervu{{\textnormal{u}}}
\def\ervv{{\textnormal{v}}}
\def\ervw{{\textnormal{w}}}
\def\ervx{{\textnormal{x}}}
\def\ervy{{\textnormal{y}}}
\def\ervz{{\textnormal{z}}}

% Random matrices
\def\rmA{{\mathbf{A}}}
\def\rmB{{\mathbf{B}}}
\def\rmC{{\mathbf{C}}}
\def\rmD{{\mathbf{D}}}
\def\rmE{{\mathbf{E}}}
\def\rmF{{\mathbf{F}}}
\def\rmG{{\mathbf{G}}}
\def\rmH{{\mathbf{H}}}
\def\rmI{{\mathbf{I}}}
\def\rmJ{{\mathbf{J}}}
\def\rmK{{\mathbf{K}}}
\def\rmL{{\mathbf{L}}}
\def\rmM{{\mathbf{M}}}
\def\rmN{{\mathbf{N}}}
\def\rmO{{\mathbf{O}}}
\def\rmP{{\mathbf{P}}}
\def\rmQ{{\mathbf{Q}}}
\def\rmR{{\mathbf{R}}}
\def\rmS{{\mathbf{S}}}
\def\rmT{{\mathbf{T}}}
\def\rmU{{\mathbf{U}}}
\def\rmV{{\mathbf{V}}}
\def\rmW{{\mathbf{W}}}
\def\rmX{{\mathbf{X}}}
\def\rmY{{\mathbf{Y}}}
\def\rmZ{{\mathbf{Z}}}

% Elements of random matrices
\def\ermA{{\textnormal{A}}}
\def\ermB{{\textnormal{B}}}
\def\ermC{{\textnormal{C}}}
\def\ermD{{\textnormal{D}}}
\def\ermE{{\textnormal{E}}}
\def\ermF{{\textnormal{F}}}
\def\ermG{{\textnormal{G}}}
\def\ermH{{\textnormal{H}}}
\def\ermI{{\textnormal{I}}}
\def\ermJ{{\textnormal{J}}}
\def\ermK{{\textnormal{K}}}
\def\ermL{{\textnormal{L}}}
\def\ermM{{\textnormal{M}}}
\def\ermN{{\textnormal{N}}}
\def\ermO{{\textnormal{O}}}
\def\ermP{{\textnormal{P}}}
\def\ermQ{{\textnormal{Q}}}
\def\ermR{{\textnormal{R}}}
\def\ermS{{\textnormal{S}}}
\def\ermT{{\textnormal{T}}}
\def\ermU{{\textnormal{U}}}
\def\ermV{{\textnormal{V}}}
\def\ermW{{\textnormal{W}}}
\def\ermX{{\textnormal{X}}}
\def\ermY{{\textnormal{Y}}}
\def\ermZ{{\textnormal{Z}}}

% Vectors
\def\vzero{{\bm{0}}}
\def\vone{{\bm{1}}}
\def\vmu{{\bm{\mu}}}
\def\vtheta{{\bm{\theta}}}
\def\va{{\bm{a}}}
\def\vb{{\bm{b}}}
\def\vc{{\bm{c}}}
\def\vd{{\bm{d}}}
\def\ve{{\bm{e}}}
\def\vf{{\bm{f}}}
\def\vg{{\bm{g}}}
\def\vh{{\bm{h}}}
\def\vi{{\bm{i}}}
\def\vj{{\bm{j}}}
\def\vk{{\bm{k}}}
\def\vl{{\bm{l}}}
\def\vm{{\bm{m}}}
\def\vn{{\bm{n}}}
\def\vo{{\bm{o}}}
\def\vp{{\bm{p}}}
\def\vq{{\bm{q}}}
\def\vr{{\bm{r}}}
\def\vs{{\bm{s}}}
\def\vt{{\bm{t}}}
\def\vu{{\bm{u}}}
\def\vv{{\bm{v}}}
\def\vw{{\bm{w}}}
\def\vx{{\bm{x}}}
\def\vy{{\bm{y}}}
\def\vz{{\bm{z}}}

% Elements of vectors
\def\evalpha{{\alpha}}
\def\evbeta{{\beta}}
\def\evepsilon{{\epsilon}}
\def\evlambda{{\lambda}}
\def\evomega{{\omega}}
\def\evmu{{\mu}}
\def\evpsi{{\psi}}
\def\evsigma{{\sigma}}
\def\evtheta{{\theta}}
\def\eva{{a}}
\def\evb{{b}}
\def\evc{{c}}
\def\evd{{d}}
\def\eve{{e}}
\def\evf{{f}}
\def\evg{{g}}
\def\evh{{h}}
\def\evi{{i}}
\def\evj{{j}}
\def\evk{{k}}
\def\evl{{l}}
\def\evm{{m}}
\def\evn{{n}}
\def\evo{{o}}
\def\evp{{p}}
\def\evq{{q}}
\def\evr{{r}}
\def\evs{{s}}
\def\evt{{t}}
\def\evu{{u}}
\def\evv{{v}}
\def\evw{{w}}
\def\evx{{x}}
\def\evy{{y}}
\def\evz{{z}}

% Matrix
\def\mA{{\bm{A}}}
\def\mB{{\bm{B}}}
\def\mC{{\bm{C}}}
\def\mD{{\bm{D}}}
\def\mE{{\bm{E}}}
\def\mF{{\bm{F}}}
\def\mG{{\bm{G}}}
\def\mH{{\bm{H}}}
\def\mI{{\bm{I}}}
\def\mJ{{\bm{J}}}
\def\mK{{\bm{K}}}
\def\mL{{\bm{L}}}
\def\mM{{\bm{M}}}
\def\mN{{\bm{N}}}
\def\mO{{\bm{O}}}
\def\mP{{\bm{P}}}
\def\mQ{{\bm{Q}}}
\def\mR{{\bm{R}}}
\def\mS{{\bm{S}}}
\def\mT{{\bm{T}}}
\def\mU{{\bm{U}}}
\def\mV{{\bm{V}}}
\def\mW{{\bm{W}}}
\def\mX{{\bm{X}}}
\def\mY{{\bm{Y}}}
\def\mZ{{\bm{Z}}}
\def\mBeta{{\bm{\beta}}}
\def\mPhi{{\bm{\Phi}}}
\def\mLambda{{\bm{\Lambda}}}
\def\mSigma{{\bm{\Sigma}}}

% Tensor
\DeclareMathAlphabet{\mathsfit}{\encodingdefault}{\sfdefault}{m}{sl}
\SetMathAlphabet{\mathsfit}{bold}{\encodingdefault}{\sfdefault}{bx}{n}
\newcommand{\tens}[1]{\bm{\mathsfit{#1}}}
\def\tA{{\tens{A}}}
\def\tB{{\tens{B}}}
\def\tC{{\tens{C}}}
\def\tD{{\tens{D}}}
\def\tE{{\tens{E}}}
\def\tF{{\tens{F}}}
\def\tG{{\tens{G}}}
\def\tH{{\tens{H}}}
\def\tI{{\tens{I}}}
\def\tJ{{\tens{J}}}
\def\tK{{\tens{K}}}
\def\tL{{\tens{L}}}
\def\tM{{\tens{M}}}
\def\tN{{\tens{N}}}
\def\tO{{\tens{O}}}
\def\tP{{\tens{P}}}
\def\tQ{{\tens{Q}}}
\def\tR{{\tens{R}}}
\def\tS{{\tens{S}}}
\def\tT{{\tens{T}}}
\def\tU{{\tens{U}}}
\def\tV{{\tens{V}}}
\def\tW{{\tens{W}}}
\def\tX{{\tens{X}}}
\def\tY{{\tens{Y}}}
\def\tZ{{\tens{Z}}}


% Graph
\def\gA{{\mathcal{A}}}
\def\gB{{\mathcal{B}}}
\def\gC{{\mathcal{C}}}
\def\gD{{\mathcal{D}}}
\def\gE{{\mathcal{E}}}
\def\gF{{\mathcal{F}}}
\def\gG{{\mathcal{G}}}
\def\gH{{\mathcal{H}}}
\def\gI{{\mathcal{I}}}
\def\gJ{{\mathcal{J}}}
\def\gK{{\mathcal{K}}}
\def\gL{{\mathcal{L}}}
\def\gM{{\mathcal{M}}}
\def\gN{{\mathcal{N}}}
\def\gO{{\mathcal{O}}}
\def\gP{{\mathcal{P}}}
\def\gQ{{\mathcal{Q}}}
\def\gR{{\mathcal{R}}}
\def\gS{{\mathcal{S}}}
\def\gT{{\mathcal{T}}}
\def\gU{{\mathcal{U}}}
\def\gV{{\mathcal{V}}}
\def\gW{{\mathcal{W}}}
\def\gX{{\mathcal{X}}}
\def\gY{{\mathcal{Y}}}
\def\gZ{{\mathcal{Z}}}

% Sets
\def\sA{{\mathbb{A}}}
\def\sB{{\mathbb{B}}}
\def\sC{{\mathbb{C}}}
\def\sD{{\mathbb{D}}}
% Don't use a set called E, because this would be the same as our symbol
% for expectation.
\def\sF{{\mathbb{F}}}
\def\sG{{\mathbb{G}}}
\def\sH{{\mathbb{H}}}
\def\sI{{\mathbb{I}}}
\def\sJ{{\mathbb{J}}}
\def\sK{{\mathbb{K}}}
\def\sL{{\mathbb{L}}}
\def\sM{{\mathbb{M}}}
\def\sN{{\mathbb{N}}}
\def\sO{{\mathbb{O}}}
\def\sP{{\mathbb{P}}}
\def\sQ{{\mathbb{Q}}}
\def\sR{{\mathbb{R}}}
\def\sS{{\mathbb{S}}}
\def\sT{{\mathbb{T}}}
\def\sU{{\mathbb{U}}}
\def\sV{{\mathbb{V}}}
\def\sW{{\mathbb{W}}}
\def\sX{{\mathbb{X}}}
\def\sY{{\mathbb{Y}}}
\def\sZ{{\mathbb{Z}}}

% Entries of a matrix
\def\emLambda{{\Lambda}}
\def\emA{{A}}
\def\emB{{B}}
\def\emC{{C}}
\def\emD{{D}}
\def\emE{{E}}
\def\emF{{F}}
\def\emG{{G}}
\def\emH{{H}}
\def\emI{{I}}
\def\emJ{{J}}
\def\emK{{K}}
\def\emL{{L}}
\def\emM{{M}}
\def\emN{{N}}
\def\emO{{O}}
\def\emP{{P}}
\def\emQ{{Q}}
\def\emR{{R}}
\def\emS{{S}}
\def\emT{{T}}
\def\emU{{U}}
\def\emV{{V}}
\def\emW{{W}}
\def\emX{{X}}
\def\emY{{Y}}
\def\emZ{{Z}}
\def\emSigma{{\Sigma}}

% entries of a tensor
% Same font as tensor, without \bm wrapper
\newcommand{\etens}[1]{\mathsfit{#1}}
\def\etLambda{{\etens{\Lambda}}}
\def\etA{{\etens{A}}}
\def\etB{{\etens{B}}}
\def\etC{{\etens{C}}}
\def\etD{{\etens{D}}}
\def\etE{{\etens{E}}}
\def\etF{{\etens{F}}}
\def\etG{{\etens{G}}}
\def\etH{{\etens{H}}}
\def\etI{{\etens{I}}}
\def\etJ{{\etens{J}}}
\def\etK{{\etens{K}}}
\def\etL{{\etens{L}}}
\def\etM{{\etens{M}}}
\def\etN{{\etens{N}}}
\def\etO{{\etens{O}}}
\def\etP{{\etens{P}}}
\def\etQ{{\etens{Q}}}
\def\etR{{\etens{R}}}
\def\etS{{\etens{S}}}
\def\etT{{\etens{T}}}
\def\etU{{\etens{U}}}
\def\etV{{\etens{V}}}
\def\etW{{\etens{W}}}
\def\etX{{\etens{X}}}
\def\etY{{\etens{Y}}}
\def\etZ{{\etens{Z}}}

% The true underlying data generating distribution
\newcommand{\pdata}{p_{\rm{data}}}
% The empirical distribution defined by the training set
\newcommand{\ptrain}{\hat{p}_{\rm{data}}}
\newcommand{\Ptrain}{\hat{P}_{\rm{data}}}
% The model distribution
\newcommand{\pmodel}{p_{\rm{model}}}
\newcommand{\Pmodel}{P_{\rm{model}}}
\newcommand{\ptildemodel}{\tilde{p}_{\rm{model}}}
% Stochastic autoencoder distributions
\newcommand{\pencode}{p_{\rm{encoder}}}
\newcommand{\pdecode}{p_{\rm{decoder}}}
\newcommand{\precons}{p_{\rm{reconstruct}}}

\newcommand{\laplace}{\mathrm{Laplace}} % Laplace distribution

\newcommand{\E}{\mathbb{E}}
\newcommand{\Ls}{\mathcal{L}}
\newcommand{\R}{\mathbb{R}}
\newcommand{\emp}{\tilde{p}}
\newcommand{\lr}{\alpha}
\newcommand{\reg}{\lambda}
\newcommand{\rect}{\mathrm{rectifier}}
\newcommand{\softmax}{\mathrm{softmax}}
\newcommand{\sigmoid}{\sigma}
\newcommand{\softplus}{\zeta}
\newcommand{\KL}{D_{\mathrm{KL}}}
\newcommand{\Var}{\mathrm{Var}}
\newcommand{\standarderror}{\mathrm{SE}}
\newcommand{\Cov}{\mathrm{Cov}}
% Wolfram Mathworld says $L^2$ is for function spaces and $\ell^2$ is for vectors
% But then they seem to use $L^2$ for vectors throughout the site, and so does
% wikipedia.
\newcommand{\normlzero}{L^0}
\newcommand{\normlone}{L^1}
\newcommand{\normltwo}{L^2}
\newcommand{\normlp}{L^p}
\newcommand{\normmax}{L^\infty}

\newcommand{\parents}{Pa} % See usage in notation.tex. Chosen to match Daphne's book.

\DeclareMathOperator*{\argmax}{arg\,max}
\DeclareMathOperator*{\argmin}{arg\,min}

\DeclareMathOperator{\sign}{sign}
\DeclareMathOperator{\Tr}{Tr}
\let\ab\allowbreak


\usepackage{hyperref}
\usepackage{url}

%%%%%% My packages %%%%%%%%%%%%%%%%%%%%%%%%%%%%%%%
\usepackage{tikz}
\usepackage{graphicx}
\usetikzlibrary{external}
\tikzexternalize % Activate externalization

    
% Settings for Tikz Plots
\usepackage{float}
\usepackage[utf8]{inputenc}
\usepackage{pgfplots}
\DeclareUnicodeCharacter{2212}{−}
\usepgfplotslibrary{groupplots,dateplot}
\usetikzlibrary{patterns, shapes.arrows}
\pgfplotsset{compat=newest}

% Extra Packages for temp use
\usepackage{lipsum}

%%%%%%%%%%%%%%%%%%%%%%%%%%%%%%%%%%%%%%%%%%%%%%%%



\title{KKT Nets: A Physics-Inspired Neural Network approach to solving convex optimization problems}

% Authors must not appear in the submitted version. They should be hidden
% as long as the \iclrfinalcopy macro remains commented out below.
% Non-anonymous submissions will be rejected without review.

\author{Rishabh Sharad Pomaje \\
Department of Electrical Engineering\\
Indian Institute Of Technology Dharwad,\\
India.\\
\texttt{\href{mailto:210020036@iitdh.ac.in}{210020036@iitdh.ac.in}} \\
\And
Rajshekhar V. Bhat\\
Department of Electrical Engineering \\
Indian Institute of Technology Dharwad, 
India.\\
\texttt{\href{mailto:rajshekhar.bhat@iitdh.ac.in}{rajshekhar.bhat@iitdh.ac.in}} \\
\AND
Coauthor \\
Affiliation \\
Address \\
\texttt{email}
}

% The \author macro works with any number of authors. There are two commands
% used to separate the names and addresses of multiple authors: \And and \AND.
%
% Using \And between authors leaves it to \LaTeX{} to determine where to break
% the lines. Using \AND forces a linebreak at that point. So, if \LaTeX{}
% puts 3 of 4 authors names on the first line, and the last on the second
% line, try using \AND instead of \And before the third author name.

\newcommand{\fix}{\marginpar{FIX}}
\newcommand{\new}{\marginpar{NEW}}

\iclrfinalcopy % Uncomment for camera-ready version, but NOT for submission.
\begin{document}

\maketitle

\begin{abstract}
   \lipsum[15]
\end{abstract}

\section{Introduction}
The general form of a convex optimization problem, \cite{Boyd_Vandenberghe_2004}, is expressed as:
\begin{subequations}
\begin{align}
\min_{\vx \in \mathbb{R}^n} \quad & f_0(\vx) \\
\text{subject to} \quad & f_i(\vx)  \leq 0, \quad i = 1, \ldots, m \\
& g_i(\vx) = 0, \quad i = 1, \ldots, p
\end{align}
\end{subequations}
where $f_i: \mathbb{R}^n \rightarrow \mathbb{R}$ are convex functions and  $g_i: \mathbb{R}^n \rightarrow \mathbb{R}$ are affine. The domain of the above problem is defined as 
\begin{align}
   \mathcal{D} = \bigcap_{i=0}^{m} \text{\textbf{dom}}f_i\ \cap\ \bigcap_{i=1}^{p} \text{\textbf{dom}}g_i\ 
\end{align}

A dual formulation of the above optimization problem is made using \emph{Lagrange's Dual Function and Multipliers}. Let $\lambda \in \mathbb{R}^m$ and $\nu \in \mathbb{R}^p$. 

The \emph{Lagrangian} $\mathcal{L}: \mathbb{R}^n \times \mathbb{R}^m \times \mathbb{R}^p \rightarrow \mathbb{R}$ is defined as, 
\begin{align}
   \mathcal{L}(\vx, \lambda, \nu) = f_0(\vx) + \sum_{i=1}^{m} \lambda_i f_i(\vx) + \sum_{i=1}^{p} \nu_i g_i(\vx).
\end{align} 
The \emph{Lagrange Dual function} (LDF) is given as,
\begin{align}
   g(\lambda, \nu) = \inf_{\vx \in \mathcal{D}} \mathcal{L}(\vx, \lambda, \nu)
\end{align}

Note that by definition the LDF is a pointwise infinimum of affine functions of $\lambda$ and $\nu$. Thus, the LDF is always a convex function regardless of the nature of the objective and constraint functions.  

Suppose \( \lambda_i \) are the dual variables associated with the inequality constraints \( f_i(x) \) for \( i \in \{1,2,\ldots,m\} \), and \( \nu_i \) are the dual variables associated with the equality constraints \( g_i(x) \) for \( i \in \{1,2,\ldots,p\} \). Under some regularity conditions, the following Karush-Kuhn-Tucker (KKT) conditions provide necessary and sufficient conditions for optimality in such problems:
\begin{align}
&\text{Primal feasibility: } f_i(x^*) \leq 0, \quad g_i(x^*) = 0, \\
&\text{Dual feasibility: } \lambda_i^* \geq 0, \\
&\text{Complementary slackness: } \lambda_i^* f_i(x^*) = 0, \\
&\text{Stationarity: } \nabla f_0(x^*) + \sum_{i=1}^{M} \lambda_i^* \nabla f_i(x^*) + \sum_{i=1}^{K} \nu_i^* \nabla g_i(x^*) = 0,
\end{align}
 
In this article, we focus on training a deep learning model to take problem parameters for specific subclasses of convex optimization problems, such as linear or quadratic programs with up to a maximum \( n \), and output the optimal primal variable \( x^* \) and dual variables \( \{ \lambda_i^* \}_{i=1}^M \) and \( \{ \nu_i^* \}_{i=1}^K \).  We propose an architecture that embeds the KKT conditions for optimality into the neural network and defines loss functions accordingly. We refer to these networks as KKT Networks.

\section{Neural Network Approach to Solving KKT Conditions}

We propose using a neural network to solve the KKT conditions by learning the optimal primal and dual variables. The neural network takes problem parameters as inputs and outputs the optimal primal variables \( x^* \) and dual variables \( \lambda^* \). The loss function is designed to capture the KKT conditions:
\begin{align}
\text{Loss} = \text{Primal Feasibility Loss} + \text{Complementary Slackness Loss} + \text{Stationarity Loss}
\end{align}
where:
\begin{align}
\text{Primal Feasibility Loss} &= \frac{1}{N} \sum_{i=1}^{M} \max(0, f_i(x))^2 \\
\text{Complementary Slackness Loss} &= \frac{1}{N} \sum_{i=1}^{M} (\lambda_i f_i(x))^2 \\
\text{Stationarity Loss} &= \frac{1}{N} \left\| \nabla f_0(x) + \sum_{i=1}^{M} \lambda_i \nabla f_i(x) + \sum_{i=1}^{K} \nu_i \nabla g_i(x) \right\|^2
\end{align}


\section{Examples of standard forms of programming}
\subsection{Linear Programming}
Standard form of Linear Program:
\begin{subequations}
   \begin{align}
      \min_{\vx \in \mathbb{R}^n} \quad & \vc^T \vx \\
      \text{subject to} \quad & \mG\vx \preceq \vh \\
      & \mA\vx = \vb
      \end{align}
\end{subequations}

\subsection{Quadratic Programming}
Standard form for Quadratic Programming:
\begin{subequations}
   \begin{align}
      \min_{\vx \in \mathbb{R}^n} \quad &\frac{1}{2}{\vx}^T \mP \vx + \vq^T \vx + r \\ 
      \text{subject to} \quad & \mG\vx \preceq \vh  \\
      & \mA\vx = \vb
   \end{align}
\end{subequations}

\subsection{Other standard problems}
There are other standard problems that categorize themselves under convex optimization such as Quadratic Programming Quadratic Constraints (QCQP), Second Order Cone Programming (SOCP), etc. We consider them as future work.

\section{System Methodology}

\subsection{Data Set Generation}
There are two classes of optimization problems. One where there is no explicit formulation and instead we access the objective and constraint functions which are modelled as a black box. We can \emph{query} this black box and get the value of the objective, or derivative of the objective and so on at any point. However, we do not have the closed form expression of any function involved. Such a formulation style is called \emph{Oracle problem description}. We will not address these type of problems in this work and consider it as future work.

In other cases where optimization problems which can be expressed in explicit, closed-form expressions are called \emph{parameterized problems}. 

For training a neural network, it is imperative that we require labelled data. We can artificially generate the required data, consisting of parameters of a problem instance and the corresponding solution. To achieve this, we can use random number generators to populate the parameters of any optimization problem. Furthermore to find the optimal primal and dual solutions, one can use any one of the numerous solvers available. Specifically for our data, we used the CVXPY python module coupled with OSQP solver. 

For instance, the standard form of a QP is as follows:
\begin{subequations}
   \begin{align}
      \min_{\vx} \quad &\frac{1}{2}\vx^T \mP \vx + \vq^T \vx + r \\
      \text{subject to} \quad &\mG \vx \preceq \vh \\
      \quad &\mA \vx = \vb  
   \end{align}
\end{subequations}
where, \[\mP \in \mathbb{R}^{n \times n}, \vq \in \mathbb{R}^{n}, r \in \mathbb{R}, \mG \in \mathbb{R}^{m \times n}, \vh \in \mathbb{R}^{m}, \mA \in \mathbb{R}^{p \times n}, \vb \in \mathbb{R}^{p}.\]

We use random number generator to populate the entries of the matrices and the vectors (of parameters) in the above expression. However, there is one issue; any vector space $V$ is defined over a field $\mathcal{F}$. If we consider a vector space over the field of real numbers, $\mathbb{R}$, we will never be able to generate a data set that covers any fraction of the entire vector space as, $\forall\ x \in \mathbb{R}, -\infty \leq x \leq \infty$. 

To address this issue, we propose a simple Complete Problem Normalization (CPN) that transforms the any original problem and limits the ``problem space'' meaning it ensures that all entries of the parameter matrices and the parameter vectors are limited to the interval [-1, 1] and that by a simple scaling factor, we can ``unnormalize'' the problem to get back the original problem and optimal solution from the normalized ones.

\subsubsection{Complete Problem Normalization (CPN)}
   Let $\Theta$ be defined as,
   \[
   \Theta = \max\{P_{max}, q_{max}, r, G_{max}, h_{max}, A_{max}, b_{max}\}.
   \]
   where each of the element of the set above is the maximum of absolute values of the entries of a matrix or a vector they belong to. For example,
   \[
   G_{max} \overset{\Delta}{=} \max\{|g_{ij}|| g_{ij} \text{is the element}\ i^{th} \text{row and } j^{th} \text{column}\}.
   \]  
   We then do the complete normalization as,
   \begin{subequations}
      \begin{align}
         \tilde{\mP} &= \mP / \Theta, \tilde{\vq} = \vq / \Theta, \tilde{r} = r / \Theta \\
         \tilde{\mG} &= \mG / \Theta, \tilde{\vh} = \vh / \Theta\\ 
         \tilde{\mA} &= \mA / \Theta, \tilde{\vb} = \vb / \Theta
      \end{align}
   \end{subequations}
   Thus, we see that a CPNed problem will be of the form: problem merely a positively scaled version and that the limits of the entries are as mentioned before. 
   
   \subsubsection{Why is CPN required?}
   As mentioned above is not possible to generate a dataset that can cover any non-zero fraction of the problem space. Thus if we generate problem instances from the problem space mentioned above then we can solve any other optimization problem.  
   
   \subsubsection{Solving problems outside the unit problem space using CPN}
   
   Let us consider the same problem. The CPNed problem will be of the form: 
   \begin{subequations}
      \begin{align}
         \min_{\vx} \quad &\frac{1}{2}\vx^T \tilde{\mP} \vx + \tilde{\vq}^T \vx + \tilde{r} \\
         \text{subject to} \quad &\tilde{\mG} \vx \preceq \tilde{\vh} \\
         \quad &\tilde{\mA} \vx = \tilde{\vb}  
      \end{align}
   \end{subequations}
   If the solution to the normalized problem is $\tilde{\vx}^*$, then due to the uniform scaling, we get the solution to the original problem as, \[
   {\vx}^* = \tilde{\vx}^*.
   \]

   \subsubsection{A study of the dataset}
   

\section{Experiments, Results and Discussion}  

\bibliographystyle{iclr2025_conference}
\bibliography{references}

\appendix
\section{Appendix}

\end{document}
